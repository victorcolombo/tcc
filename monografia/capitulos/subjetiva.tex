\chapter{Parte Subjetiva}
\label{subjetiva}


A minha primeira memória de um computador é de quando eu tinha 8 anos de idade. Era 2005, meu pai entrou em casa com um computador munido do poderoso Pentium III, com Windows XP instalado e um jogo, naquela época distribuído em CD-ROM.

Ao ver o computador inicializar pela primeira vez, sabia que era aquilo que eu queria fazer, todo dia, pelo resto da minha vida. Para mim, porém, não bastava jogar, navegar na internet e mandar e-mails: eu sonhava em entender como aquela \quotes{caixa preta} funcionava de verdade.

Foram muitas tentativas de me aprofundar no assunto. Depois de abandonar dois cursos técnicos, ser recusado de um curso profissionalizante por ser muito novo e me frustrar com tutoriais na internet, percebi que todos tentaram me ensinar \quotes{como fazer} mas não \quotes{porque funciona}. Assim, ficou claro que precisava buscar ensino superior.

Na faculdade, porém, não seria tão simples quanto eu imaginava. As aulas e os livros muitas vezes não eram suficientes para internalizar completamente os conceitos.

Como um jeito de me aprofundar e continuar evoluindo em algoritmos e estruturas de dados, entrei para o grupo de extensão de estudo para competições de programação, o \href{https://www.ime.usp.br/~maratona/}{MaratonIME}.

Durante minha participação nessas competições, tive a chance de conhecer e aprender com pessoas muito talentosas, viajar dentro e fora do Brasil, e desenvolver uma capacidade de resolução de problemas que raramente é explorada nos cursos básicos da graduação.

Uma dificuldade, porém, persistia: algoritmos gulosos, o objeto de estudo deste trabalho. Vendo que a dificuldade neste tópico era presente  desde estudantes iniciantes no curso até competidores experientes, decidi usar minha experiência de algoritmos \quotes{mão na massa} como base da escrita para produzir este material didático.

Espero que tenha sido capaz de transmitir meu conhecimento e contribuído para que mais pessoas consigam aprender computação de maneira mais didática e interessante.