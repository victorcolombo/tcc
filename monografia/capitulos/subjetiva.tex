\chapter{Parte Subjetiva}
\label{subjetiva}


A minha primeira memória é de quando eu tinha 8 anos de idade. Era 2005, meu pai entrou em casa com um computador munido do poderoso Pentium III, com Windows XP instalado e um jogo, naquela época distribuído em CD-ROM.

Ao ver o computador inicializar pela primneira vez, sabia que era aquilo que eu queria fazer, todo dia, pelo resto da minha vida. Para mim, porém, não bastava jogar, navegar na internet e mandar e-mails: eu sonhava em entender como aquela "caixa preta" funcionava de verdade.

Foram muitas tentativas de me aprofundar no assunto. Depois de abandonar dois cursos técnicos, ser recusado de um curso profissionalizante por ser muito novo e me frustrar com tutoriais na internet, percebi que todos tentaram me ensinar "como fazer" mas não "por que funciona". Asssim, ficou claro que precisava buscar ensino superior.

Na faculdade, porém, não seria tão simples quanto eu imaginava, já que a falta de didática dos livros e cursos abarrotados de conteúdos me faziam, muitas vezes, decorar conceitos sem realmente entendê-los. Uma das maiores dificuldades foi, sem dúvida, aprender algoritmos gulosos, o objeto de estudo deste trabalho.

Através da minha participação em competições de programação competitiva, consegui aprofundar meu conhecimento no tema e desenvolver uma capacidade de resolução de problemas que raramente é explorada nos cursos básicos da graduação. Usei esta experiência de algoritmos "mão na massa" como base da escrita.  

Minha insatisfação com o modo com que este tópico é passado, aliada com minha vontade de entender computação num nível fundamental e tendo passado dificuldades para aprendê-lo sozinho, contribuiu para que eu tenha escrito este material didático.