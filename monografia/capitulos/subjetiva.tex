\chapter{Parte Subjetiva}
\label{subjetiva}


A minha primeira memória é de quando eu tinha 8 anos de idade. Era 2005, meu pai entrou em casa com um computador munido do poderoso Pentium III, com Windows XP instalado e um jogo, naquela época distribuído em CD-ROM.

Ao ver o computador inicializar pela primneira vez, sabia que era aquilo que eu queria fazer, todo dia, pelo resto da minha vida. Para mim, porém, não bastava jogar, navegar na internet e mandar e-mails: eu sonhava em entender como aquela "caixa preta" funcionava de verdade.

Foram muitas tentativas de me aprofundar no assunto. Depois de abandonar dois cursos técnicos, ser recusado de um curso profissionalizante por ser muito novo e me frustrar com tutoriais na internet, percebi que todos tentaram me ensinar "como fazer" mas não "por que funciona". Assim, ficou claro que precisava buscar ensino superior.

Na faculdade, porém, não seria tão simples quanto eu imaginava. As aulas e os livros muitas vezes não eram suficientes para internalizar completamente os conceitos. Uma das maiores dificuldades foi, sem dúvida, aprender algoritmos gulosos, o objeto de estudo deste trabalho.

Através da minha participação em competições de programação competitiva, com o apoio do grupo de extensão MaratonIME, consegui aprofundar meu conhecimento no tema e desenvolver uma capacidade de resolução de problemas que raramente é explorada nos cursos básicos da graduação. Usei essa experiência de algoritmos "mão na massa" como base da escrita.  

Após me sentir confortável com os conceitos que um dia me tiraram o sono, procurei, durante toda a graduação, transmitir o conhecimento através de aulas, videoaulas, cursos, apostilas e, por fim, este TCC. Espero que tenha contribuído para que mais pessoas consigam aprender computação de maneira mais didática e interessante.