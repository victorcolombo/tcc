\chapter{Subproblemas gulosos}
\label{subproblema}

Nos capítulos anteriores, foram abordados problemas cuja única resolução era a aplicação de um critério guloso, seja de escolha ou ordenação. Neste capítulo, o problema apresentado não terá solução gulosa mas, através de aplicação de outras técnicas, podemos reduzir o problema a um subproblema que aceita solução gulosa.

\section{Operações em vetor}

É dado um vetor de inteiros $h = \{h_1, h_2, ..., h_n\}$ e dois inteiros $X \geq 0$ e $0 \leq D \leq n - 1$. Você pode fazer até $X$ operações em $h$. Cada operação é definida como incrementar em $1$ todos $h_j$, tal que $j \in [i, min(i + D, n)]$, isto é, $h_i \rec h_i + 1$, $h_{i + 1} \rec h_{i + 1} + 1$, ..., $h_{min(i + D, n)} \rec h_{min(i + D, n)} + 1$.

Queremos encontrar o maior $H$ tal que o $h_i \geq H$ para qualquer $i \in [1, n]$, após a aplicação de $X$ operações em quaisquer intervalos de tamanho até $D$.

\subsection*{Exemplos}

\begin{enumerate}
    \item TODO
\end{enumerate}

\section{Desenvolvimento}

Imediatamente conseguimos perceber que este problema tem muitas \quotes{partes soltas}, dificultando a aplicação direta de um algoritmo guloso. Precisamos maximizar $H$, que depende do $h$ final, que por sua vez depende das operações escolhidas.

Uma abordagem útil quando estamos estagnados é modificar o problema para outro que parece mais simples e, depois de resolvê-lo, tentar generalizar ou modificar a solução para resolver o problema original. Em particular, vamos fixar a variável resposta $H$.

Suponha que $H$ é fixo. Gostaríamos de saber se é possível transformar $h$ num vetor tal que $H$ é menor ou igual a todos seus elementos após realizar até $X$ operações. Note que transformamos um problema de otimização em um problema binário.

\subsubsection*{Algoritmo guloso para $H$ fixo}\label{subproblema:algo}

A partir daqui conseguimos ter algumas ideias gulosas. Primeiramente precisamos modificar $h_1$ para que $h_1 \geq H$.

Se, inicialmente, $h_1 \geq H$, não gostaríamos de \quotes{gastar} nenhuma operação em algum intervalo que contém $h_1$. Se $h_1 < H$, precisamos pelo menos $H - h_1$ operações para que $h_1$ satisfaça a condição. Note que somente o intervalo $[1, min(1 + D, n)]$ contém o índice $1$, nos poupando da escolha do intervalo.

Depois de \quotes{arrumar} $h_1$, repetimos o mesmo processo para $h_2$, $h_3$ e assim por diante. Ao \quotes{arrumar} todos os elementos de $h$, se o número de operações aplicadas forem menores ou iguais $X$, $H$ é uma resposta viável.

\subsubsection*{Maximizando $H$}

Agora que já sabemos como resolver o problema binário para $H$ fixo, como achamos o maior $H$ possível?

Uma ideia ingênua seria iterar sobre todos os valores possíveis de $H$. É fácil encontrar o universo de valores que $H$ pode assumir: no mínimo, nenhuma operação é realizada e $H$ é o valor do menor elemento do vetor $h$. No máximo, todas as $X$operações são aplicadas em intervalos contendo o menor elemento de $h$. Assim, basta testarmos os valores de $H$ no intervalo $[min(h), min(h) + X]$ ($min(h)$ sendo o elemento mínimo do vetor $h$). Tal abordagem nos daria uma solução não polinomial, pois a complexidade de tempo seria proporcional a $X$.

A observação essencial é que se um valor $H$ é viável, todo $H' < H$ também é. Assim, podemos reduzir o espaço de busca pela metade a cada iteração, similar à uma Busca Binária. Esta abordagem nos dá uma solução polinomial proporcional a $O(\lg X)$.

\subsection*{Demonstrações}

Provaremos as afirmações feitas na seção anterior. Utilizaremos $h^*$ como notação para o vetor $h$ após a aplicação de um conjunto de operações.

Definiremos $H$ como \textbf{viável} se existe um conjunto de até $X$ operações que aplicadas a $h$ produzem um vetor $h^*$ tal que $h^*_i \geq H$ para todo $i \in [1, n]$.

\begin{prop}
    $H$ é viável $\so$ $H^*$ é viável para todo $H^* < H$
\end{prop}

\begin{proof}
    Seja $h^*$ o vetor resultante da aplicação de $k \leq X$ operações necessárias para que $h^*_i \geq H$ para todo $i \in [1, n]$. É trivial que $h^*_i \geq H - 1$. Assim temos que a aplicação das mesmas operações que produzem um vetor $h^*$ no qual $H - 1$ é viável.
\end{proof}

\begin{prop}
    Algoritmo descrito em \ref{subproblema:algo} cria um vetor $h^*$ que respeita a restrição $h^*_i \geq H$ utilizando o menor número de operações possível.
\end{prop}

\begin{proof}
    Definimos uma sequência de operações $s$ como uma sequência de índices ordenados que representam operações num intervalo com tal índice como extremidade esquerda. Por exemplo: $s = \{1, 3\}$ é a aplicação das operações que incrementa valores de $h$ de índices $[1, min(1 + D, n)]$ e $[3, min(3 + D, n)]$.
    
    Seja $s$ a sequência de operações aplicadas que criam vetor $h^*$ que respeita a restrição de $H$. e seja $s'$ a sequência de operações mínima com maior prefixo comum com $s$ que cria um vetor $h'$ que respeita a restrição de $H$.
    
    Se $s = s'$, está provado.

    Caso contrário, seja $k$ o menor índice tal que $s_k \neq s'_k$.

    Pela invariante do algoritmo vale que, se $s_i \neq s_{i - 1}$, todos os $h^*_j = H$ para todo $j < s_i$, ou seja, todos os índices antes de $s_i$ já satisfazem a condição. Também vale que, para todo índice $s_i$, $h^*_{s_i}$ é justo, isto é, $h^*_{s_i} = H$.

    Se $s_k < s'_k$, $s'$ possui uma ocorrência a menos de $s_k$ que $s$ e $h^*_j = H$, então $h'_j < H$, uma contradição na hipótese que $h'$ cria um vetor que respeita a restrição de $H$.

    Se $s_k > s'_k$, $s'$ possui uma ocorrência a mais de $s'_k$ que $s$. Ora mas como $h^*_{s'_k} = H$, então $h'$ apresenta folga, ou seja, $h'_{s'_k} > H$. Assim, podemos substituir $s'_k$ por $s_k$ em $s'$, fazendo com que $h'_{s'_k} = H$ e mantendo o mesmo ou aumentando os $h'_j$ com $j > s'_k$. Esta é uma contradição na hipótese que $s'$ tem máximo prefixo comum com $s$.

\end{proof}

Como o algoritmo \ref{subproblema:algo} utiliza o menor número de operações, se este número for menor ou igual a $X$, significa que o $H$ fixado é fazível.

\section{Implementação e análise}

\begin{algorithm}[H]
\caption{Solução para o Problema \ref{subproblema}}
\label{subproblema:code}
\begin{algorithmic}[1]
\Function{\textsc{Valido}}{h, n, X, D, H}
    \For{$i$ de $1$ até $n$}
        \State $dif \rec max(H - h[i], 0)$
        \State $X \rec X - dif$
        \For{$j$ de $i$ até $min(i + D, n)$}
            \State $h[j] \rec h[j] + dif$
        \EndFor
    \EndFor
    \If{$X \geq 0$}
        \State \Return True
    \Else
        \State \Return False
    \EndIf
\EndFunction
\Function{\textsc{Resolve}}{h, n, X, D}
    \State $esq \rec min(h)$
    \State $dir \rec min(h) + X$
    \While{$esq < dir$}
        \State $meio \rec \frac{esq + dir}{2}$
        \If{$\textsc{Valido}(h, n, X, D, meio)$}
            \State $esq \rec meio$
        \Else
            \State $dir \rec meio - 1$
        \EndIf
    \EndWhile
    \State \Return esq
\EndFunction
\end{algorithmic}
\end{algorithm}

No pior caso, a função \textsc{Valido} aplica exatamente uma operação num intervalo com extremidade em cada elemento. Cada operação gasta $O(D)$ para atualizar os elementos de $h$ afetados. Assim, a complexidade de \textsc{Valido} é $O(nD)$.

A função \textsc{Resolve}, por sua vez, chama \textsc{Valido} e reduz o espaço de busca de $H$ na metade a cada iteração. Assim, a complexidade de \textsc{Resolve} é $O(nD \lg X)$.

Embora a solução seja polinomial pois $D$ é limitado por $n$, ainda há como melhorar a solução. Podemos desconfiar que \textsc{Valido} tenha espaço para otimizações. 

Precisamos de uma estrutura de dados que suporta incrementar intervalo e acessar posições do vetor de forma eficiente. Uma opção é utilizar uma Árvore de Segmentos com Propagação Preguiçosa \cite{mehta2004handbook}, reduzindo a complexidade de cada operação de $O(D)$ para $O(\lg n)$, resultando na complexidade $O(n \lg n \lg X)$.

Podemos atingir uma complexidade ainda melhor que a abordagem anterior sem a utilização de nenhuma estrutura de dados sofisticada. A observação essencial é estamos iterando nos vetor de maneira sequencial e todas as operações estão sendo feitas em ordem crescente da extremidade esquerda. Deste fato podemos utilizar uma ideia semelhante a Propagação Preguiçosa: manteremos um vetor que indica quantas operações terminam em cada índice. Com isso, basta manter acumulado, a cada iteração, quantas operações passaram por um elemento, modificando tal variável de acordo com as operações iniciadas e terminadas no índice atual.

\begin{algorithm}[H]
\caption{Função \textsc{Valido} em tempo linear}
\label{subproblema:code_linear}
\begin{algorithmic}[1]
\Function{\textsc{ValidoLinear}}{h, n, X, D, H}
    \State $preguica \rec \{0\}^{n + 1}$
    \State $acum \rec 0$
    \For{$i$ de $1$ até $n$}
        \State $dif \rec max(H - h[i] + acum, 0)$
        \State $X \rec X - dif$
        \State $acum \rec dif + preguica[i]$
        \State $preguica[min(i + D, n + 1)] \rec preguica[min(i + D, n + 1)] + 1$
    \EndFor
    \If{$X \geq 0$}
        \State \Return True
    \Else
        \State \Return False
    \EndIf
\EndFunction
\end{algorithmic}
\end{algorithm}

É fácil ver que \textsc{ValidoLinear} é $O(n)$. Substituindo \textsc{Valido} por \textsc{ValidoLinear} em \ref{subproblema:code} chegamos na complexidade de tempo $O(n \lg X)$ e memória $O(n)$.
