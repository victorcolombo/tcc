\chapter{Argumento de troca}
\label{salto}

É muito comum que problemas que aceitam soluções gulosas possuam diversas outras soluções ótimas. Assim, demonstrações por contradição que assumem a existência de uma solução ótima que é diferente da solução gulosa são insuficientes, já que solução escolhida também pode ser ótima, mas distinta da gulosa.

Visando superar esta dificuldade, podemos aplicar uma técnica conhecida como \quotes{argumento de troca}. Esta consiste em escolher uma solução ótima conveniente. Tomamos uma solução ótima \quotes{mais parecida} com a solução gulosa possível. Isto é, escolhemos uma solução ótima com maior prefixo de escolhas em comum com a solução gulosa. Desta forma, se a solução gulosa for ótima, elas coincidirão.

Buscamos a contradição assumindo que as escolhas diferem em algum momento. Daí buscamos \quotes{trocar} a decisão da solução ótima pela solução gulosa e mostramos que tal troca não altera o resultado, contrariando a hipótese do maior prefixo de escolhas em comum.

A seguir aplicaremos tal técnica para demonstrar a corretude da solução para um problema.

\section{Salto do sapo}

Existem pedras $n$ pedras numa reta numérica, em posições distintas $v_1, v_2, ..., v_{n - 1}, v_n$. Dizemos que o sapo pode saltar de uma pedra $v_i$ para outra pedra $v_j$ desde que a distância entre elas seja menor ou igual a $\Delta$. Um sapo está inicialmente na pedra $v_1$. Qual é o menor número de saltos que ele precisa dar para chegar na pedra $v_n$? 

Alternativamente, é dado um vetor de $n$ números \underline{ordenados distintos} $v = \{v_1, v_2, ..., v_n\}$ e um número $\Delta$.

Uma sequência $u = \{u_1, u_2, ..., u_k\}$ é \textbf{solução} se:
\begin{itemize}
    \item $u_1 = v_1$
    \item $u_k = v_n$
    \item $u_i = v_j$ para todo $i \in [1, k]$ e algum $j \in [1, n]$
    \item $|u_i - u_{i + 1}| \leq \Delta$ para $i \in [1, k - 1]$
\end{itemize}

Vamos supor que sempre existe pelo menos um $u$.

Desejamos encontrar uma sequência $u$ que satisfaça as propriedades acima e que o tamanho $k$ de $u$ seja mínimo.

Chamamos este $u$ de solução ótima para o problema.

\subsection*{Exemplos}

\begin{enumerate}
    \item A entrada $n = 4$, $v = \{1, 2, 3, 4\}$, $\Delta = 1$ existem diversas soluções possíveis, entre elas $\{1, 2, 3, 4\}$, $\{1, 2, 1, 2, 3, 4\}$ e $\{1, 2, 1, 2, 3, 2, 3, 4\}$. A sequência de menor $k = 4$ é $u = \{1, 2, 3, 4\}$.
    \item A entrada $n = 6$, $v = \{1, 2, 3, 5, 6, 7\}$, $\Delta = 2$ tem como solução ótima $u = \{1, 3, 5, 7\}$.
    \item A entrada $n = 3$, $v = \{1, 3, 4\}$, $\Delta = 1$ não admite solução, já que a partir de $1$ não é possível que $3$ ou $4$ sejam o próximo elemento da sequência. Como dissemos, desconsideraremos estes casos neste problema.
\end{enumerate}

\section{Desenvolvimento}

Primeiramente devemos notar que nunca vale a pena \textit{\quotes{voltar}}, isto é, escolher um número menor que o escolhido anteriormente, pois isto aumentaria desnecessariamente a sequência, já que poderíamos descartar a escolha anterior e escolher apenas o menor número.

Além disso, devemos fazer a observação que, para todo $y \in v$, nenhum número maior que $y$ pode ser sucessor de algum número menor que $y$ sem que pudesse ser sucessor do próprio $y$. Ou seja, intuitivamente não existe \textit{\quotes{vantagem}} em escolher um número menor que o maior possível.

%\subsection*{Abordagem}
\label{salto:abor}

Partindo das observações anteriores, uma solução intuitiva é \textit{\quotes{ir mais para direita possível}}: Partindo de $u_1 = v_1$, escolha o próximo $u_{i + 1}$ tal que a diferença ao elemento anterior, $u_{i + 1} - u_i$, seja máxima e menor ou igual a $\Delta$, repetindo até escolher $v_n$.

Neste caso, felizmente, o primeiro algoritmo considerado é de fato o correto. Agora o trabalho é reduzido a formalizar e demonstrar sua corretude.

\subsection*{Demonstrações}

\begin{prop} \label{salto:cres}
Se $u$ é ótima, $u$ é estritamente crescente.
\end{prop}
\begin{proof}

Se $k = 1$, está provado.

Se $k = 2$, $u$ é estritamente crescente e ótima por definição, já que $u_1 = v_1 < v_n = u_{k}$

Suponha que $k \geq 3$ e que $u$ é ótima mas não é crescente. Seja $i$ o primeiro índice tal que $u_{i} > u_{i + 1}$. Da definição, segue que:

\begin{equation} \label{eq:diffr}
0 \leq u_i - u_{i + 1} \leq \Delta
\end{equation}

Além disso, temos que $\{u_1, u_2, ..., u_i\}$ é crescente. Ou seja, vale que $u_{i - 1} < u_i$. Da definição, segue que:
\begin{equation} \label{eq:diffl}
0 \leq u_i - u_{i - 1} \leq \Delta
\end{equation}

Multiplicando a equação \ref{eq:diffr} por -1 e somando com \ref{eq:diffl} temos:
\begin{align}  \label{eq:diff}
-\Delta \leq - u_i + u_{i + 1} \leq 0 \nonumber \\
0 \leq u_i - u_{i - 1} \leq \Delta \nonumber \\
-\Delta \leq u_{i + 1} - u_{i - 1} \leq \Delta \nonumber \\
|u_{i + 1} - u_{i - 1}| \leq \Delta
\end{align}

Da equação \ref{eq:diff} segue que $\bar{u} = \{u_1, u_2, ..., u_{i - 1}, u_{i + 1}, ..., u_k\}$ é solução de tamanho $k - 1$. Contradição, já que por hipótese $k$ é mínimo.

\end{proof}

\begin{theo} \label{salto:proof}
O algoritmo \ref{salto:code} produz uma solução ótima.
\end{theo}
\begin{proof}

Suponha que o algoritmo proposto produz uma solução $u$ de tamanho $k$. Sabemos pela proposição anterior que $u$ é estritamente crescente.

Seja $u^*$ de tamanho $k^*$ uma solução ótima com maior prefixo comum com $u$. Ou seja, se $u^*_i = u_i$ para $i \in [1, l]$, $u^*$ é tal que $l$ é máximo. Como $u^*$ é ótima, temos que $k^* \leq k$. Sabemos pela proposição \ref{salto:cres} que $u^*$ é estritamente crescente.

Provaremos por contradição que $k^* = k$.  Suponha que $k^* < k$.

Se $l = k^*$, temos uma contradição, pois como $u^*_{k^*} = u^*_l = v_n$ e $u^*_l = u_l$, então $u_l = u_k$, contrariando a condição de parada do algoritmo. 

Se $l < k^*$, temos que $u^*_{l} = u_{l}$, $u^*_{l + 1} \neq u_{l + 1}$. Pelo critério de escolha do algoritmo, temos que $u_{l + 1} > u^*_{l + 1}$.

Caso 1: $l + 1 = k^*$. Como $u^*_{l + 1} = v_n$ e $v_n$ é o maior elemento de $v$, não existe escolha tal que $u_{l + 1} > v_n$. Contradição.

Caso 2: $l + 1 < k^*$. Como as sequências $u$ e $u^*$ são válidas, valem as seguintes identidades:

$$u^*_{l + 2} - u^*_{l + 1} \leq \Delta$$
$$u^*_{l + 1} - u^*_{l} \leq \Delta$$
$$u_{l + 1} - u_{l} \leq \Delta$$

Também, como tanto $u$ quanto $u^*$ são estritamente crescentes, $u_{l + 1} > u^*_{l + 1}$, concluímos que $|u^*_{l + 2} - u_{l + 1}| \leq \Delta$.

Desta forma, podemos \textbf{trocar} $u^*_{l + 1}$ por $u_{l + 1}$, criando a sequência $\bar{u} = \{u_1, u_2, ..., u_l, u_{l + 1}, u^*_{l + 2}, ..., u^*_{l^*}\}$. Como $\bar{u}$ tem o mesmo tamanho que $u^*$ e possui prefixo comum com $u$ de tamanho $l + 1$, temos uma contradição na hipótese que $l$ é máximo.

\end{proof}

\section{Implementação e análise}

A ideia descrita na seção anterior é simples de se implementar, basta manter o último $v_i$ selecionado para $u$ e atualizá-lo assim que um $v_j$ exceder a distância $\Delta$.

\begin{algorithm}[H]
\caption{Solução gulosa para o Problema \ref{salto}}
\label{salto:code}
\begin{algorithmic}[1]
\Function{\textsc{Resolve}}{v, n, \Delta}
    \State $u_1 \rec v_1$
    \State $j \rec 2$
    \For{$i$ de $2$ até $n$}
        \If{$v_i - u_j > \Delta$}
            \State $u_j \rec v_{i - 1}$
            \State $j \rec j + 1$
        \EndIf
    \EndFor
    \State $u_j \rec v_n$
    \State \Return $u$
\EndFunction
\end{algorithmic}
\end{algorithm}

É fácil ver que o pseudocódigo acima tem complexidade de tempo $O(n)$ e de memória $O(1)$.

\section{Exercícios}

\subsection*{Teóricos}

\begin{enumerate}
  \item Modifique o algoritmo para que ele trate quando a instância não possui resposta. Prove que sua modificação não produz uma solução se e somente se não há uma solução.
  \item Suponha que $v$ possui inteiros repetidos e não necessariamente em ordem. É possível generalizar a solução? Demonstre.
\end{enumerate}

\subsection*{Problemas}

\begin{enumerate}
  \setcounter{enumi}{2}
  \item \href{https://codeforces.com/contest/946/problem/A}{Partition} - 
Educational Codeforces Round 39 (Rated for Div. 2)
  \item \href{https://codeforces.com/problemset/problem/946/C}{String Transformation} - Educational Codeforces Round 39 (Rated for Div. 2)
  \item \href{https://icpcarchive.ecs.baylor.edu/index.php?option=com_onlinejudge&Itemid=8&page=show_problem&problem=4356}{The Glittering Caves of Aglarond} - Asia Amritapuri Regional Contest 2012
  \item \href{https://codeforces.com/contest/903/problem/B}{The Modcrab} - Educational Codeforces Round 34 (Rated for Div. 2)
\end{enumerate}

