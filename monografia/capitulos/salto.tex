\chapter{Argumento de troca}
\label{salto}

\subsection*{Introdução}

É muito comum que problemas que aceitam soluções gulosas possuam diversas outras soluções ótimas. Assim, demonstrações por contradição que assumem a existência de uma solução ótima que é diferente da solução gulosa são insuficientes, já a solução gulosa também pode ser ótima, mas distinta da escolhida.

Visando superar esta dificuldade, podemos aplicar uma técnica conhecida como "argumento de troca". Esta consiste em escolher uma solução ótima conveniente. Tomamos uma solução ótima "mais parecida" com a solução gulosa possível. Isto é, escolhemos uma solução ótima com maior prefixo de escolhas em comum com a solução gulosa. Desta forma, se a solução gulosa for ótima, elas coincidirão.

Buscamos a contradição assumindo que as escolhas diferem em algum momento. Daí buscamos "trocar" a decisão da solução ótima pela solução gulosa e mostrando que tal troca não altera o resultado, contrariando a hipótese do maior prefixo de escolhas em comum.

A seguir aplicaremos tal técnica para demonstrar a corretude de um problema.

\subsection*{Enunciado}

É dado um vetor de $n$ números ordenados distintos $v = \{v_1, v_2, ..., v_n\}$ e um número $\Delta$.

Uma sequência $u = \{u_1, u_2, ..., u_k\}$ é \textbf{solução} se:
\begin{itemize}
    \item $u_1 = v_1$
    \item $u_k = v_n$
    \item $u_i = v_j$ para todo $i \in [1, k]$ e algum $j \in [1, n]$
    \item $|u_i - u_{i + 1}| \leq \Delta$ para $i \in [1, k - 1]$
\end{itemize}

Vamos supor que sempre existe pelo menos um $u$.

Desejamos encontrar uma \textbf{solução ótima}: um sequência $u$ que é solução tal que o tamanho de $u$ é mínimo.

Diremos que $u$ é ótima se é uma solução ótima.

\subsection*{Exemplos}

\begin{enumerate}
    \item A entrada $n = 4$, $v = \{1, 2, 3, 4\}$, $\Delta = 1$ existem diversas sequências possíveis, entre elas $\{1, 2, 3, 4\}$, $\{1, 2, 1, 2, 3, 4\}$ e $\{1, 2, 1, 2, 3, 2, 3, 4\}$. A sequência de menor $k = 4$, $u = \{1, 2, 3, 4\}$.
    \item A entrada $n = 6$, $v = \{1, 2, 3, 5, 6, 7\}$, $\Delta = 2$ tem como solução $u = \{1, 3, 5, 7\}$.
    \item A entrada $n = 3$, $v = \{1, 3, 4\}$, $\Delta = 1$ não apresenta solução, já que a partir de $1$ não é possível que $3$ ou $4$ sejam o próximo elemento da sequência. Não consideraremos este caso neste problema.
\end{enumerate}

\subsection*{Observações iniciais}

Primeiramente devemos notar que nunca vale a pena \textit{"voltar"}, isto é, escolher um número menor que o escolhido anteriormente, pois isto aumentaria desnecessariamente a sequência, já que poderíamos descartar a escolha anterior e escolher apenas o menor número.

Além disso, devemos fazer a observação que, para todo $y \in v$, nenhum número maior que $y$ pode ser sucessor de algum número menor que $y$ sem que pudesse ser sucessor do próprio $y$. Ou seja, intuitivamente não existe \textit{"vantagem"} em escolher um número menor que o maior possível.

\subsection*{Abordagem}
\label{salto:abor}

Partindo das observações anteriores, uma solução intuitiva é \textit{"ir mais para direita possível"}: Partindo de $u_1 = v_1$, escolha o próximo $u_{i + 1}$ tal que a diferença ao elemento anterior, $u_{i + 1} - u_i$, seja máxima e menor que $\Delta$, repetindo até escolher $v_n$.

Neste caso, felizmente, o primeiro algoritmo considerado é de fato o correto. Agora o trabalho é reduzido a formalizar e demonstrar sua corretude.

\subsection*{Formalização}

Vamos descrever precisamente as ideias exploradas na secção anterior através de um pseudocódigo.

\begin{algorithm}[h]
\caption{Solução gulosa para o Problema \ref{salto}}
\label{salto:code}
\begin{algorithmic}[1]
\Function{\textsc{Resolve}}{v, n, \Delta}
    \State $u_1 \rec v_1$
    \State $j \rec 2$
    \For{$i$ de $2$ até $n$}
        \If{$v_i - u_j > \Delta$}
            \State $u_j \rec v_{i - 1}$
            \State $j \rec j + 1$
        \EndIf
    \EndFor
    \State $u_j \rec v_n$
    \State \Return u
\EndFunction
\end{algorithmic}
\end{algorithm}

\subsection*{Demonstração}

\begin{prop} \label{salto:cres}
Se $u$ é ótima, $u$ é estritamente crescente.
\end{prop}
\begin{proof}

Se $k = 1$, está provado.

Se $k = 2$, $u$ é estritamente crescente e ótima por definição, já que $u_1 = v_1 < v_n = u_{k}$

Suponha que $k \geq 3$ e que $u$ é ótima mas não é crescente. Seja $i$ o primeiro índice tal que $u_{i} > u_{i + 1}$. Da definição, segue que:

\begin{equation} \label{eq:diffr}
0 \leq u_i - u_{i + 1} \leq \Delta
\end{equation}

Além disso, temos que $\{u_1, u_2, ..., u_i\}$ é crescente. Ou seja, vale que $u_{i - 1} < u_i$. Da definição, segue que:
\begin{equation} \label{eq:diffl}
0 \leq u_i - u_{i - 1} \leq \Delta
\end{equation}

Multiplicando a equação \ref{eq:diffl} por -1 e somando com \ref{eq:diffr} temos:
\begin{align}  \label{eq:diff}
-\Delta \leq - u_i + u_{i + 1} \leq 0 \nonumber \\
0 \leq u_i - u_{i - 1} \leq \Delta \nonumber \\
-\Delta \leq u_{i + 1} - u_{i - 1} \leq \Delta \nonumber \\
|u_{i + 1} - u_{i - 1}| \leq \Delta
\end{align}

Da equação \ref{eq:diff} segue que $\bar{u} = \{u_1, u_2, ..., u_{i - 1}, u_{i + 1}, ..., u_k\}$ é solução de tamanho $k - 1$. Contradição, já que por hipótese $k$ é mínimo.

\end{proof}

\begin{theo} \label{salto:proof}
O algoritmo \ref{salto:code} produz uma solução ótima.
\end{theo}
\begin{proof}

Suponha que o algoritmo proposto produz uma solução $u$ de tamanho $k$. Sabemos pelo funcionamento do algoritmo que $u$ é estritamente crescente.

Seja $u^*$ de tamanho $k^*$ uma solução ótima com maior prefixo comum com $u$. Ou seja, se $u^*_i = u_i$ para $i \in [1, x]$, $u^*$ é tal que $x$ é máximo. Como $u^*$ é ótima, temos que $k^* \leq k$. Sabemos pela proposição \ref{salto:cres} que $u^*$ é estritamente crescente.

Provaremos por contradição que $k^* = k$.  Suponha que $k^* < k$.

Se $x = k^*$, temos uma contradição, pois como $u^*_{k^*} = u^*_x = v_n$ e $u^*_x = u_x$, então $u_x = u_k$, contrariando a condição de parada do algoritmo. 

Se $x < k^*$, temos que $u^*_{x} = u_{x}$, $u^*_{x + 1} \neq u_{x + 1}$. Pelo critério de escolha do algoritmo, temos que $u_{x + 1} > u^*_{x + 1}$.

Caso 1: $x + 1 = k^*$. Como $u^*_{x + 1} = v_n$ e $v_n$ é o maior elemento de $v$, não existe escolha tal que $u_{x + 1} > v_n$. Contradição.

Caso 2: $x + 1 < k^*$. Por definição, valem as seguintes identidades:

$$u^*_{x + 2} - u^*_{x + 1} \leq \Delta$$
$$u^*_{x + 1} - u^*_{x} \leq \Delta$$
$$u_{x + 1} - u_{x} \leq \Delta$$

Como tanto $u$ quanto $u^*$ são estritamente crescentes, $u_{x + 1} > u^*_{x + 1}$, concluímos que $|u^*_{x + 2} - u_{x + 1}| \leq \Delta$.

Desta forma, podemos \textbf{trocar} $u^*_{x + 1}$ por $u_{x + 1}$, criando a sequência $\bar{u} = \{u_1, u_2, ..., u_x, u_{x + 1}, u^*_{x + 2}, ..., u^*{k^*}\}$. Como $\bar{u}$ tem o mesmo tamanho que $u^*$ e possui prefixo comum com $u$ de tamanho $x + 1$, temos uma contradição na hipótese que $x$ é máximo.

\end{proof}

Exercício 1: Modifique o algoritmo para que ele trate quando a instância não possui resposta. Prove que sua modificação não produz uma solução se e somente se não há uma solução.

Exercício 2: Suponha que $v$ possui inteiros repetidos e não necessariamente em ordem. É possível generalizar a solução? Demonstre.

