\chapter{Resumo}
\label{resumo}

Algoritmos gulosos são uma classe de algoritmos muito importantes para a resolução de problemas de otimização. Nesses algoritmos, a solução ótima globalmente é atingida através de escolhas ótimas localmente, também conhecidas como escolhas gulosas.

Embora este tópico esteja presente na ementa de qualquer curso introdutório de algoritmos, muitos alunos têm dificuldades para discernir quando utilizá-lo em detrimento das demais técnicas. Nestes cursos, normalmente são apresentados apenas problemas clássicos, como \emph{Activity selection}, com livro texto em língua estrangeira.

Foi produzido um material didático em português focado no ensino de técnicas por meio de resolução de problemas desafiantes e não usuais, como vistos em competições como Maratona de Programação e ACM ICPC, a fim de desenvolver no leitor a capacidade de identificar se um problema tem estratégia gulosa ótima, propor tal solução e implementá-la, transformando este processo em algo sistemático.


\textbf{Palavras-chave:} algoritmos gulosos, programação competitiva, material didático, otimização.
