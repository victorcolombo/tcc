\chapter{Explorando restrições}
\label{particao}

\section{Problema da Partição}

É dado um vetor de $n$ números inteiros, $V = \{v_1, v_2, ..., v_n\}$, tal que, para todo $i \in [1, n]$, vale que $1 \leq v_i \leq i$.

O objetivo é encontrar, se existir, uma partição de $v$ em dois conjuntos de igual soma.

Alternativamente, queremos encontrar $r = \{r_1, r_2, ..., r_n\}$, com $r_i \in \{-1, 1\}$ para todo $i \in [1, n]$, tal que:
$$\sum_{i = 1}^n v_i*r_i = 0$$

\subsection*{Exemplos}

\begin{enumerate}
    \item A entrada $n = 4, V = \{1, 2, 3, 3\}$ não apresenta particionamento possível pois $1 + 2 + 3 + 3 = 9$ é ímpar.
    \item A entrada $n = 4, V = \{1, 2, 3, 4\}$ apresenta o particionamento $r = \{-1, 1, 1, -1\}$. Note que o particionamento $r = \{1, -1, -1, 1\}$ também é válido, por simetria.
    \item A entrada $n = 4, V = \{1, 1, 3, 3\}$ apresenta tanto o particionamento $r = \{-1, 1, -1, 1\}$ quanto $r = \{-1, 1, 1, -1\}$, ou seja, podem existir mútiplos particionamentos válidos, não necessáriamente simétricos.
    \item A entrada $n = 3, V = \{1, 1, 3\}$ não apresenta particionamento possível, pois $v_3 = 3 > 1 + 1$.
    \item A entrada $n = 3, V = \{1, 2, 4\}$ não é uma entrada válida já que não vale a restrição $1 \leq v_3 \leq 3$.
\end{enumerate}

\section{Desenvolvimento}

Antes de resolver o problema, vale tentar estabelecer conexão com problemas famosos ou de estrutura semelhante.

O primeiro pensamento que vem à tona é a semelhança do enunciado com o Problema da Partição. Na verdade, o problema enunciado pode ser encarado como uma instância particular deste problema clássico.

Isto já nos traz um grande arsenal de informações, pois sabemos que o problema original é NP-completo \cite{karp1972reducibility} e pode ser resolvido em tempo pseudo-polinomial usando Programação Dinâmica.

Para obter algoritmos eficientes para o problema devemos, de alguma forma, explorar as restrições adicionais que foram incluídas.

Neste caso, a restrição de entrada $1 \leq v_i \leq n$ é a única diferença entre o Problema da Partição e o problema apresentado. 
Problemas NP-completo são, à primeira vista, bastante atraentes para soluções gulosas. Evidentemente, um algoritmo baseado em uma estratégia gulosa não produz uma solução ótima para todas as instâncias do problema, pois, como tais algoritmos têm complexidade polinomial, levaria a uma prova de que P = NP.

% \subsection*{Abordagens}
\label{particao:abordagem}

Um algoritmo guloso natural é, iterativamente, contruir as partições, começando com duas partições vazias, encaminhamos um elemento à partição com menor soma parcial, tentando deixar as partições mais \quotes{equilibradas} possível a todo momento, isto é, minizando a diferença entre as partições em toda iteração.

Além disso, como estamos trabalhando com restrição na entrada em função do índice, temos a intuição de que a ordem em que esses elementos são adicionados importa. Daí seguem duas opções imediatas: iterando crescentemente de $1$ a $n$ ou iterando descrescentemente de $n$ a $1$.

\subsubsection*{Iterando crescentemente}

Podemos rapidamente descartar esta ordem retornando ao exemplo 2) que já analisamos. Na entrada $V = \{1, 2, 3, 4\}$.

Encaminhamos o elemento $v_1 = 1$ para uma partição arbitrária, já que as duas estão vazias. A partir daí, encaminhamos $v_2 = 2$ para a partição oposta a de $v_1$. Aplicamos o mesmo raciocínio para $v_3$ e $v_4$.

Chegando assim na resposta inválida $r = \{-1, 1, -1, 1\}$, que resulta em partições de somas distintas, já que $(-1)*1 + (+1)*2 + (-1)*3 + (+1)*4 \neq 0$.

Esta instância mostra que o algoritmo não funciona para todos os casos.

\subsubsection*{Iterando decrescentemente}

Vamos testar com a mesma entrada $V = \{1, 2, 3, 4\}$, temos:

Encaminhamos o elemento $v_4 = 4$ para uma partição arbitrária, já que as duas estão vazias. A partir daí, encaminhamos $v_3 = 3$ para a partição oposta a de $v_4$. Por sua vez, $v_2 = 2$ é encaminhado para a partição de $v_3$. Aplicamos o mesmo raciocínio para $v_1$.

Chegando assim na resposta válida $r = \{-1, 1, 1, -1\}$, que resulta em partições de somas iguais, já que $(-1)*1 + (+1)*2 + (+1)*3 + (-1)*4 = 0$.

Mesmo que exemplos não garantam a corretude do algoritmo para todos os casos, realizá-los pode aumentar nossa confiança antes de começarmos uma demonstração formal de corretude.

Já que nosso algoritmo apresenta uma série de sucessos em exemplos que montamos, é a hora de tentar formalizá-lo.

\subsection*{Demonstração}

Sejam $A$ e $B$ as duas partições tal que dizemos que $v_i$ pertence a partição $A$ se $r_i = 1$ e pertence a $B$ se $r_i = -1$. Seguem algumas definições:

\begin{defi}
$S_i$ é a soma acumulada do prefixo $v_1, v_2, ..., v_i$ de $V$: $S_i = \sum_{j = 1}^{i} v_j$.
\end{defi}

\begin{defi}
$a_i$ é a \quotes{folga} de $A$, isto é, a diferença entre o valor esperado da partição e a soma dos elementos que já foram encaminhados a $A$, depois do processamento dos elementos $v_{i + 1}, v_{i + 2}, ..., v_{n}$ pelo algoritmo.
Analogamente, definimos $b_i$ para a partição $B$.
\end{defi}

\begin{prop} \label{particao:odd}
$S_n = \sum_{i = 1}^n$ é impar $\Rightarrow$ Não há solução.
\end{prop}

\begin{prop} \label{particao:proof}
Se $S_n$ for par, para qualquer iteração $n - i + 1$ do algoritmo, existe folga de tamanho pelo menos $v_i$ em $A$ ou $B$. Ou seja, $a_i \geq v_i$ ou $b_i \geq v_i$.
\end{prop}
\begin{proof}

Considere a iteração que decide para que partição $v_i$ será encaminhado, com as partições $A$ e $B$ com folgas $a_i$ e $b_i$, respectivamente.

Primeiramente devemos notar que
\begin{equation} \label{eq:1}
    a_i + b_i = S_i
\end{equation}
, já que todo elemento deverá ser encaminhado para alguma partição.

Pela definição, sabemos que $i \geq v_i \geq 1$. A partir disso, temos que:

$$S_{i - 1} = \sum_{j = 1}^{i - 1} v_i \geq \sum_{j = 1}^{i - 1} 1 = i - 1 \geq v_i - 1$$

Assim, podemos escrever $S_i$ como:

\begin{equation} \label{eq:2}
    S_i = S_{i - 1} + v_i \geq 2*v_i - 1
\end{equation}

Vamos assumir que nem $A$, nem $B$, tenham \quotes{folga} suficiente para acomodar $v_i$. Provaremos por contradição, separado em 2 casos:

\subsubsection*{Caso 1: $a_i$ e $b_i$ tem a mesma paridade.}

Se $a_i$ e $b_i$ tem a mesma paridade, $S_i$ é par.

Como nenhuma partição tem \quotes{folga} para acomodar $v_i$, vale que $a_i < v_i$ e $b_i < v_i$. Somando as equações, vale que $a_i + b_i < 2*v_i$ ou também $a_i + b_i \leq 2*v_i - 1$.

Utilizando \ref{eq:1} e \ref{eq:2}, segue que $a_i + b_i \geq 2*v_i - 1$.

Ora mas se $a_i + b_i \geq 2*v_i - 1$ e $a_i + b_i \leq 2*v_i - 1$, vale que:

$$S_i = a_i + b_i = 2*v_i - 1$$

Contradição, já que $2*v_i - 1$ é ímpar e $S_i$ é par.

\subsubsection*{Caso 2: $a_i$ e $b_i$ tem diferentes paridades.}

Como nenhuma partição tem \quotes{folga} para acomodar $v_i$, temos que $a_i < v_i$ e $b_i < v_i$.

Ora mas como $a_i$ e $b_i$ tem paridades diferentes, deve valer que $a_i < v_i$ e $b_i < v_i - 1$ ou $a_i < v_i - 1$ e $b_i < v_i$.

Vamos assumir, sem perda de generalidade, que $a_i < v_i$ e $b_i < v_i - 1$. Somando ambas as desigualdades, segue que $a_i + b_i < 2*v_i - 1$ ou $a_i + b_i \leq 2*v_i - 2$.

Utilizando \ref{eq:1} e \ref{eq:2}, temos que $a_i + b_i \geq 2*v_i - 1$.

Assim, se $a_i + b_i \geq 2*v_i - 1$ e $a_i + b_i \leq 2*v_i - 2$, há uma contradição, já que a intersecção é vazia.

\end{proof}

\section{Implementação e análise}

Com a proposição \ref{particao:proof} provada agora podemos encontrar um algoritmo para resolver o problema. Como mostramos que há \quotes{folga} de tamanho pelo menos $v_i$ em alguma das partições em toda iteração, basta que, a cada iteração, encaminhemos o elemento atual para uma das partições com \quotes{folga} suficiente. A seguir, descrevemos essa ideia em pseudocódigo:

\begin{algorithm}[h]
\caption{Solução gulosa para o Problema \ref{particao}}
\label{particao:code}
\begin{algorithmic}[1]
\Function{\textsc{SomaVetor}}{v, n}
    \State $soma \rec 0$
    \For{$i$ de $1$ até $n$}
        \State $soma \rec soma + v_i$
    \EndFor
    \State \Return soma
\EndFunction

\Function{\textsc{Resolve}}{v, n}
    \State $soma \rec \text{SomaVetor}(v, n)$
    \If{$soma \% 2 \neq 0$}
        \State \Return \textbf{Impossível}
    \EndIf
    \State $folgaA \rec \frac{soma}{2}$
    \State $r \rec \{0\}^n$
    \For{$i$ de $n$ até $1$}
        \If{$folgaA > v_i$}
            \State $folgaA \rec folgaA - v_i$
            \State $r_i \rec 1$
        \Else
            \State $r_i \rec -1$
        \EndIf
    \EndFor
    \State \Return $r$
\EndFunction
\end{algorithmic}
\end{algorithm}

É fácil ver que o pseudocódigo acima executa em tempo linear em $n$, utilizando memória constante.

\section{Considerações finais}

Vários corolários interessantes saem desta demonstração.

\begin{cor}
$S_n$ é par $\Leftrightarrow$ Há solução.
\end{cor}
\begin{proof}

($\Leftarrow$) Vide \ref{particao:odd}.

($\Rightarrow$) Estabelecemos na seção anterior um algoritmo que sempre encontra solução caso $S_n$ for par. Assim, $S_n$ ser par é condição suficiente para existência da solução.

\end{proof}

\begin{cor}
Se $a_i > v_i$ e $b_i > v_i$, então $v_i$ pode ser encaminhado para qualquer partição.
\end{cor}

Note que, tanto na construção do algoritmo \ref{particao:code} quanto na demonstração da proposição \ref{particao:proof}, não fizemos uso da ideia de encaminhar para a partição com menor soma parcial (ou maior folga parcial), como discutida no início da seção \hyperref[particao:abordagem]{Abordagens}. Tal ideia, porém, dá um critério para decidir qual partição o elemento deve ir, já que a maior folga parcial é sempre positiva.

No algoritmo desenvolvido, há a prioridade sempre de encaminhar o elemento para partição $A$ e, caso não houver \quotes{folga} suficiente, envia para partição $B$. Esta escolha foi, todavia, arbitrária, já que um outro algoritmo não intuitivo que respeita a proposição seria encaminhar o elemento aleatoriamente entre as partições com \quotes{folga} de tamanho pelo menos $v_i$ para todo $i$.
