\chapter{Problema da Partição}
\label{particao}

\subsection*{Enunciado}

É dado um vetor de $n$ números inteiros, $v = \{v_1, v_2, ..., v_n\}$, tal que, para todo $i \in [1, n]$, vale que $1 \leq v_i \leq i$.

O objetivo é encontrar, se existir, uma partição de $v$ em dois conjuntos de igual soma.

Formalmente, queremos encontrar $r = \{r_1, r_2, ..., r_n\}$, com $r_i \in \{-1, 1\}$ para todo $i \in [1, n]$, tal que:
$$\sum_{i = 1}^n v_i*r_i = 0$$

\subsection*{Exemplos}

\begin{enumerate}
    \item A entrada $n = 4, v = \{1, 2, 3, 3\}$ não apresenta particionamento possível pois $1 + 2 + 3 + 3 = 9$ é ímpar.
    \item A entrada $n = 4, v = \{1, 2, 3, 4\}$ apresenta o particionamento $r = \{-1, 1, 1, -1\}$. Note que o particionamento $r = \{1, -1, -1, 1\}$ também é válido, por simetria.
    \item A entrada $n = 4, v = \{1, 1, 3, 3\}$ apresenta tanto o particionamento $r = \{-1, 1, -1, 1\}$ quanto $r = \{-1, 1, 1, -1\}$, ou seja, podem existir mútiplos particionamentos válidos, não necessáriamente simétricos.
    \item A entrada $n = 3, v = \{1, 1, 3\}$ não apresenta particionamento possível, pois $v_3 = 3 > 1 + 1$.
    \item A entrada $n = 3, v = \{1, 2, 4\}$ não é uma entrada válida já que não vale a restrição $1 \leq v_3 \leq 3$.
\end{enumerate}

\subsection*{Observações iniciais}

Antes de resolver o problema, vale tentar estabelecer conexão com problemas famosos ou de estrutura semelhante.

O primeiro pensamento que vem a tona é a semelhança do enunciado com o Problema da Partição\footnote{\url{https://en.wikipedia.org/wiki/Partition_problem}}. Além disso, o problema enunciado pode ser encarada como uma instância direta deste problema clássico.

Isto já nos traz um grande arsenal de informações, pois sabemos que o problema original é NP-completo e pode ser resolvido em tempo pseudo-polinomial usando Programação Dinâmica. Agora basta pensar se podemos fazer melhor.

Outro produto da conexão com problemas NP-completo é que podemos \textit{"desconfiar"} das restrições adicionais, explorando-as em busca de uma solução melhor que do problema original.

Neste caso, a restrição de entrada $1 \leq v_i \leq n$ é a única diferença entre o Problema da Partição e o problema apresentado. 
Problemas NP-completo são, à primeira vista, bastante atraentes para soluções gulosas. Evidentemente, pela sua natureza, todas essas solução são incorretas. Com a nova restrição, porém, podemos reavaliar alguns destes algoritmos.

\subsection*{Abordagens}

Um algoritmo guloso natural é, iterativamente, contruir as partições, começando com duas partições vazias, encaminhamos um elemento à partição com menor soma, tentando deixar as partições mais \emph{"equilibradas"} possível a todo momentos, isto é, minizando a diferença entre as partições em todo momento.

Além disso, como estamos trabalhando com restrição na entrada em função do índice, temos a intuição de que a ordem em que esses elementos são adicionados importa. Daí segue duas opções imediatas: iterando crescentemente de $1$ a $n$ ou iterando descrescentemente de $n$ a $1$.

\subsubsection*{Iterando crescentemente}

Podemos rapidamente descartar esta ordem retornando ao exemplo 2) que já analisamos. Na entrada $v = \{1, 2, 3, 4\}$.

Encaminhamos o elemento $v_1 = 1$ para uma partição arbitrária, já que as duas estão vazias. A partir daí, encaminhamos $v_2 = 2$ para a partição oposta a de $v_1$. Aplicamos o mesmo raciocínio para $v_3$ e $v_4$.

Chegando assim na resposta inválida $r = \{-1, 1, -1, 1\}$, que resulta em partições de somas distintas, já que $(-1)*1 + (+1)*2 + (-1)*3 + (+1)*4 \neq 0$.

Descartamos assim esta ordem.

\subsubsection*{Iterando decrescentemente}

Uma intuição de porque esta ordem funcionaria e a anterior não é o fato de que iterando descrescentemente, o limite superior de cada elemento diminui, pois este é limitado por seu índice, ou seja, a cada passo o tamanho de cada pilha varia cada vez menos até convergir ou não até a resposta.

Testando com a mesma entrada $v = \{1, 2, 3, 4\}$, temos:

Encaminhamos o elemento $v_4 = 4$ para uma partição arbitrária, já que as duas estão vazias. A partir daí, encaminhamos $v_3 = 3$ para a partição oposta a de $v_4$. Por sua vez, $v_2 = 2$ é encaminhado para a partição de $v_3$. Aplicamos o mesmo raciocínio para $v_1$.

Chegando assim na resposta válida $r = \{-1, 1, 1, -1\}$, que resulta em partições de somas iguais, já que $(-1)*1 + (+1)*2 + (+1)*3 + (-1)*4 = 0$.

\emph{Exercício:} Este exemplo é bem simples. Você deve testar com outras entradas mais complexas para tentar desprovar o algoritmo.

Já que nosso algoritmo apresenta uma série de sucessos em exemplos que montamos, é a hora de tentar formalizá-lo.



\subsection*{Formalização e demonstração}

Sejam $A$ e $B$ as duas partições tal qiue dizemos que $v_i$ pertence a partição $A$ se $r_i = 1$ e pertence a $B$ se $r_i = -1$.

Definições:

$S_i$ é a soma acumulada do prefixo $v_1, v_2, ..., v_i$. Formalmente $S_i = \sum_{j = 1}^{i} v_j$.

$a_i$ é a "folga" de $A$, isto é, a diferença entre o valor esperado da partição e a soma dos elementos que já foram encaminhados a $A$, depois do processamento dos elementos $v_{i + 1}, v_{i + 2}, ..., v_{n}$ pelo algoritmo.
Analogamente, definimos $b_i$ para a partição $B$.

Proposição 1: Se $S_n$ é ímpar, não há solução.

Prova: Pela definição, a soma das partições $A$ e $B$ devem ser iguais para existir solução. Assim, cada partição deve ter soma $\frac{S_n}{2}$.

Suponha que $S_n$ é ímpar e há solução.

É fácil ver que $\frac{S_n}{2} \not\in \mathbb{Z}$. Absurdo, já que a soma de uma partição que é um subconjunto de $v$, com $v_i \in \mathbb{Z}$, é inteira. Por contradição, ou $S_n$ não é ímpar ou não há solução

A partir desde ponto assumiremos que $S_n$ é par.

Proposição 2: Para qualquer iteração $n - i + 1$ do algoritmo, existe folga de tamanho pelo menos $v_i$ em $A$ ou $B$. Formalmente, $a_i \geq v_i$ ou $b_i \geq v_i$

\subsubsection*{Prova:}

Considere a iteração que decide para que partição $v_i$ será encaminhado, com as partições $A$ e $B$ com folgas $a_i$ e $b_i$, respectivamente.

Primeiramente devemos notar que
\begin{equation} \label{eq:1}
    a_i + b_i = S_i
\end{equation}
, já que todo elemento deverá ser encaminhado para alguma partição.

Pela definição, sabemos que $S_{i - 1} \geq i - 1$ e $v_i \leq i$. A partir disso, temos que $S_{i - 1} \geq i - 1 \geq v_i - 1$.

Assim, podemos escrever $S_i$ como:

\begin{equation} \label{eq:2}
    S_i = S_{i - 1} + v_i \geq 2*v_i - 1
\end{equation}

Vamos assumir que nem $A$, nem $B$, tenham "folga" suficiente para acomodar $v_i$. Provaremos por contradição, separado em 2 casos:

\subsubsection*{Caso 1: $a_i$ e $b_i$ tem a mesma paridade.}

Se $a_i$ e $b_i$ tem a mesma paridade, $S_i$ é par.

Como nenhuma partição tem "folga" para acomodar $v_i$, vale que $a_i < v_i$ e $b_i < v_i$. Somando as equações, vale que:

$a_i + b_i < 2*v_i$ ou também $a_i + b_i \leq 2*v_i - 1$

Utilizando \ref{eq:1} e \ref{eq:2}, segue que $a_i + b_i \geq 2*v_i - 1$.

Ora mas se $a_i + b_i \geq 2*v_i - 1$ e $a_i + b_i \leq 2*v_i - 1$, vale que:

$$S_i = a_i + b_i = 2*v_i - 1$$

Contradição, já que $2*v_i - 1$ é ímpar e $S_i$ é par.

\subsubsection*{Caso 2: $a_i$ e $b_i$ tem diferentes paridades.}

Como nenhuma partição tem "folga" para acomodar $v_i$ e que $a_i$ e $b_i$ tem paridades diferentes, vale que $a_i < v_i$ e $b_i < v_i - 1$ ou $a_i < v_i - 1$ e $b_i < v_i$

Vamos assumir, sem perda de generalidade, que $a_i < v_i$ e $b_i < v_i - 1$.

Utilizando \ref{eq:1} e \ref{eq:2}, segue que $a_i + b_i < 2*v_i - 1$, ou também, $a_i + b_i \leq 2*v_i - 2$.

Ora mas se $a_i + b_i \geq 2*v_i - 1$ e $a_i + b_i \leq 2*v_i - 2$, há uma constradição, já que a intersecção é vazia.

\subsection*{Observações finais}

Vários colorários interessantes saem desta demonstração.

Colorário 1: $S_n$ é ímpar \textbf{se e somente se} não há solução.

Isso é uma diferença fundamental a instância genérica do problema, em que sequer decidir se o problema tem solução é NP-completo.

Colorário 2: A pilha em que o objeto é colocado não importa, desde que haja "folga".

Encaminhar elementos para a menor pilha é um modo sem ambiguidades montar as partições, mas vale notar que em nenhum momento utilizamos critério do mínimo na demonstração. Assim poderiamos criar um algoritmo que escolhe colocar o elemento atual na pilha $A$ e somente se não houvesse "folga" colocasse em $B$. Outro algoritmo não intuitivo seria encaminhar o elemento aleatoriamente entre as partições com folga.

\subsection*{Análise e implementação}

Verificar se a soma é par é facilmente resolvida com um iterar em todos os elementos, mantendo a soma total. Complexidade O(n).

Para encontrar a partição basta iterar em todos os elementos de maneira reversa, mantendo dois contadores representando as "folgas", do mesmo jeito que foi feito na demonstração. Complexidade O(n).

ADICIONAR CÓDIGO
